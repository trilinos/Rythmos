\cleardoublepage{}

\section{Introduction}
\begin{quote}
\textbf{Notice:} Once again before beginning this report, we would
like to bring the readers attention to the fact that this document
has been released in an ``\textbf{as-is}'' condition, and therefore
is in an incomplete state and has many readability issues. However
it is hoped that the information contained in this document will still
be helpful to users in their application development.
\end{quote}
Here we design and describe a set of C++ interfaces and concrete implementations
for the solution of a broad class of transient ordinary differential
equations (ODEs) and differential algebraic equations (DAEs) in a
consistent manner. 

\cleardoublepage{}

\section{Mathematical Formulation of DAEs/ODEs for Basic Time Steppers\label{rythmos:sec:mathformulation}}

Here we describe the basic mathematical form of a general nonlinear
DAE (or ODE) for the purpose of presenting it to a forward time integrator.
At the most general level of abstraction, we will consider the solution
of fully implicit DAEs of the form 
\begin{eqnarray}
f(\dot{x},x,t) & = & 0,\;\mbox{for}\;t\in[t_{0},t_{f}],\label{rythmos:eq:dae}\\
x(t_{0}) & = & x_{0}\label{rythmos:eq:dae:ic}
\end{eqnarray}
\begin{tabbing} where\hspace{5ex}\= \\
 \> $x\:\in\:\mathcal{X}$ is the vector of differential state variables,
\\
 \> $\dot{x}=d(x)/d(t)\:\in\:\mathcal{X}$ is the vector of temporal
derivatives of $x$, \\
 \> $t,t_{0},t_{f}\:\in\:\RE$ are the current, initial, and the
final times respectively, \\
 \> $f(\dot{x},x,t)\:\in\:\mathcal{X}^{2}\times\RE\rightarrow\mathcal{F}$
defines the DAE vector function, \\
 \> $\mathcal{X}\:\subseteq\:\RE^{n_{x}}$ is the vector space of
the state variables $x$, and \\
 \> $\mathcal{F}\:\subseteq\:\RE^{n_{x}}$ is the vector space of
the output of the DAE function $f(\ldots)$. \end{tabbing}

Here we have been careful to define the Hilbert vector spaces $\mathcal{X}$
and $\mathcal{F}$ for the involved vectors and vector functions.
The relevance of defining these vector spaces is that they come equipped
with a definition of the space's scalar product (i.e.\ $<u,v>_{\mathcal{X}}$
for $u,v\in\mathcal{X}$) which should be considered and used when
designing the numerical algorithms.

The above general DAE can be specialized to more specific types of
problems based on the nature of Jacobian matrices $\jac{f}{\dot{x}}\in\mathcal{F}|\mathcal{X}$
and $\jac{f}{x}\in\mathcal{F}|\mathcal{X}$. Here we use the notation
$\mathcal{F}|\mathcal{X}$ to define a linear operator space that
maps vectors from the vector space $\mathcal{X}$ to the vector space
$\mathcal{F}$. Note that the adjoint of such linear operators are
defined in terms of these vector spaces (i.e.\ $<Au,v>_{\mathcal{F}}=<A^{T}v,u>_{\mathcal{X}}$
where $A\in\mathcal{F}|\mathcal{X}$ and $A^{T}$ denotes the adjoint
operator for $A$). Here we assume that these first derivatives exist
for the specific intervals of $t\in[t_{0},t_{f}]$ of which such an
time integrator algorithm will be directly applied the the problem.

The precise class of the problem is primarily determined by the nature
of the Jacobian $\jac{f}{\dot{x}}$: 
\begin{itemize}
\item $\jac{f}{\dot{x}}=I\in\mathcal{F}|\mathcal{X}$ yields an explicit
ODE 
\item $\jac{f}{\dot{x}}$ full rank yields an implicit ODE 
\item $\jac{f}{\dot{x}}$ rank deficient yields a general DAE. 
\end{itemize}
In addition, the ODE/DAE may be linear or nonlinear and DAEs are classified
by their index \cite{BCP}. It is expected that a DAE will be able
to tell a integrator what type of problem it is (\emph{i.e.}, explicit
ODE, implicit ODE, general DAE) and which, if any, of the variables
are linear in the problem. This type of information can be exploited
in a time integration algorithm.

Another formulation we will consider is the semi-explicit DAE formulation:
\begin{equation}
\begin{array}{rcl}
\dot{y} & = & f(y,z,t)\;\mbox{for}\;t\in[t_{0},t_{f}],\\
0 & = & g(y,z,t)\;\mbox{where}\;x=[y,z],\\
x(t_{0}) & = & x_{0}.
\end{array}\label{rythmos:eq:dae:semiexplicit}
\end{equation}

For each transient computation, the formulation will be cast into
the general form in (\ref{rythmos:eq:dae})\textendash (\ref{rythmos:eq:dae:ic})
to demonstrate how this general formulation can be exploited in many
different contexts. 

It is important for DAE initial-value problems that the initial values
of the solution, $x_{0}$, and time derivative, $\dot{x}_{0}$, satisfy
the DAE residual\cite{Brown1998}
\[
f(\dot{x}_{0},x_{0},t_{0})=0
\]

